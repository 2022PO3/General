\section{Design motivation}\label{sec:design-motivation}
The design discussed in this report was not chosen without careful consideration of all options. The first decision made, was which micro controller to use. Common types of micro controllers such as a Raspberry Pi or an Arduino are readily available and are suitable for the project's needs. The Raspberry Pi 3B was used because the department had it available. \\

Secondly, the detection of entering or leaving cars can be done by cameras with motion detection software. This means that the camera is constantly running and detects whether or not there is any movement. The positive side of this method is that there is no need for any sensors or other extra hardware. The downside is that it might be too much to handle for the Raspberry Pi. An alternative option is working with sensors (e.g. a distance sensor) to detect the cars. This is less heavy for the Raspberry Pi, but adds an extra cost to the garage. Another option would be to leave the cameras filming and run the algorithm directly on the video footage. But this may also be too much to handle for the Raspberry Pi. Therefore, the option with the sensors is chosen for this project. \\

Thirdly, the detection of available parking spaces can be done by several sensors. The most useful ones are \ac{udms} or light sensors. The latter is more expensive and does not offer any extra advantages over the \ac{udms}. For this purpose the \acp{udms} (\textsc{hc-sr04}) are used in this project.\\

Fourthly, The backend application is written in Django, an open-source web framework, written in Python. The eventual decision was made based on the following concerns: 1) Python is known to all group members; 2) Django is a very explicit framework, which does not include a lot of magic features like Ruby on Rails does; 3) Django describes itself as the ``framework for perfectionists with deadlines'', which is exactly suited for the job \cite{django_website}. \\

Lastly, the frontend application will be written in Dart, with the Flutter framework of Google. Other valid alternatives were primarily JavaScript (\ac{js})-frameworks (e.g. React or AngularJS). The main benefit for Flutter
over the other frameworks is that it can run on any operation system (Android, iOS, MacOS, Linux, Windows,
etc.), that it provides type safety and null safety (Dart is a strongly typed language) and that it supports
hot reloads, which makes development much easier [Flutter, 2022]. Furthermore, two of the team members
already worked with Flutter. \\

Of course the price of the different components played also a big part in the decision making. The prices of the individual components and the total price can be found in Table \ref{tab:budget}. The leftmost column gives the name of the component used in the
model. The second column shows how many pieces of that component are needed. Column 3 shows the price per piece and column 4 the total price for a specific component.
The budget for this project is 250 euros. The design uses a total of 224.87 euros, displayed in the second to last row of Table \ref{tab:budget}. This indicates that the limit of the budget is nearly reached with a surplus of
25.13 euros. \newline


\begin{table}[htp]
    \centering
    \caption{Final budget overview.}
\begin{tabular}{|c|c|c|c|}
	\hline
	\textbf{Component} & \textbf{Amount} & \textbf{Price/piece} & \textbf{Total} \\
	\hline
	\textsc{dorhea} Raspberry Pi Mini Camera & 2 & 11.95 & 23.9 \\
	\hline
	Ultrasonic Module Distance & 8 & 3.95 & 31.6 \\
	\hline
	\textsc{mdf} plates 6mm & 3 & 2.4 & 7.2 \\
	\hline
	Green \textsc{led} lights & 6 & 0.35 & 2.1 \\
	\hline
	Red \textsc{led} lights & 6 & 0.33 & 1.98 \\
	\hline
	Resistors & 12 & 0.2 & 2.4 \\
	\hline
	Raspberry Pi extension cable & 2 & 4.99 & 9.98 \\
	\hline
	Micro Servo Motor & 2 & 7.21 & 14.42 \\
	\hline
    Raspberry Pi 3B V1.2 & 2 & 59.95 & 119.9 \\
    \hline
    LCD-Screen & 1 & 1.49 & 1.49 \\
    \hline
    Jumper cables (20 pieces) & 2 & 4.95 & 9.9 \\
    \hline
	\multicolumn{2}{|c|}{\textbf{Total Price}} & \multicolumn{2}{c|}{224.87} \\
	\hline
	\multicolumn{2}{|c|}{\textbf{Remaining}} & \multicolumn{2}{c|}{25.13} \\

	\hline
\end{tabular}
        
    \label{tab:budget}
\end{table}
