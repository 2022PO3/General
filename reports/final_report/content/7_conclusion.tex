\section{Conclusion}\label{sec:conclusion}
The main goal of this project was to design an \ac{iot}-infrastructure which makes parking easier, faster and foremost safer. The main difficulty of designing this infrastructure, was the requirement of it being a safe system to park a car without violating the users privacy. \\

The \ac{ips} proves to be a safe and privacy-friendly way of parking, which is easy to use, both for regular users and garage owners. The main system consists out of three main parts: the local garage system, a frontend application and a backend.

\ind The local garage system is controlled by two \ac{iot}-devices, which collect data from multiple electronic sensors in the parking garage. Another thing that these devices provide, is the communication with the backend. The current obstacles for the local garage system are cable wiring and the lack of a reliable \ac{lan}-network

\ind The backend -- deployed on a single server -- mainly provides a secure way to store and retrieve user information in a privacy friendly way. Besides that, the backend implements multiple measures to enhance overall security. High latency, as well as being vulnerable to severe \ac{ddos} attacks proved to be a challenge for the current implementation of the backend.

\ind The frontend application is the main entry point for end-users to interact with the backend. With a multitude of flows, it provides a clear and comprehensible way for users to create reservations, view information about their current park and to pay the bills of the parking garage. The lacking of a proper cashing system introduces lag in the application. \\

The current prototype of the entire system has to be improved and optimised in order to scale to a large number of user requests with in-memory database, client-side cashing and a dedicated \ac{lan}-network in the local garage system.




\begin{comment}
The final design exists out of the frontend, backend and a miniature prototype of a physical parking garage. Whereas the frontend exists out of a website and mobile application that is written with Flutter. To use these applications it is not necessary to have an account or having the app installed on the user's mobile device, which enhances ease of use. Next, the backend provides communication between the frontend application and the parking garage. For the purpose of this project it ran on a local computer. Finally the miniature design of the parking garage consists of \ac{mdf}-plates and electronic equipment, namely cameras -- needed for license plate recognition --, \acp{udms} and \acp{led}. \\

The main difficulty and priority in creating this system, was its safety. The first problem that had to be solved was with cybersecurity in the frontend. To make sure that a users account could not be easily hacked, was making sure that the user had a strong password and that he/she enabled \ac{2fa}. In the backend there were challenges to make sure that not everyone was able to access the data of different users or other vulnerable data that is stored in the backend. This issue was solved by implementing tokens. Of course there were many other problems that had to be solved, but one of the biggest problems was how to prove that a license plate belonged to one unique account. In reality this could be solved by uploading a registration certificate of the car.
\end{comment}