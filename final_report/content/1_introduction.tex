\section{Introduction}\label{Introduction}

% Brede context + versmalling
People these days value efficiency in life on all fronts ever higher. Time has become more and more valuable and loss of time has to be minimal. Since almost every family owns a car with most of them even owning two or three \cite{vehicles_in_families}, and travelling by car has become increasingly popular. Parking efficiency is an element that affects everyone on a near daily basis. This requires engineers to invent and examine new solutions to improve this parking experience. The dissatisfaction with the classic ticket system in most parking garages, causes an evolution into more digital and automated systems. Often accomplished through licence plate recognition in combination with mobile apps and automated payment options \cite{4411}. Despite the convenience of these systems, it poses a risk of a privacy breach \cite{privacy_breach}. Therefore, sufficient attention should be given to the security of these systems. The privacy of the user is the number one priority.\\

% Doel
The main goal of the project is to make an automatic parking garage using \ac{iot} devices. This means that a client will be able to drive into the garage and park his/her car here for a certain duration of time. Then, drive away without having to pay with the use of a ticket. This is accomplished by cameras and \ac{anpr} software, together with a mobile application. \\

This report will describe both the mechanical and software aspects of the design. Section \ref{sec:general design} describes the working of the entire system in an abstract way. Section \ref{sec:implementation} gives the concrete implementation of the system. The reasons why the components of the system were chosen can be found in section \ref{sec:design-motivation}. Then, section \ref{sec:case-example} will describe a sample user experience in the \ac{ips}. Next, section \ref{sec:system analysis} gives a critical analysis of the final system. Section \ref{sec:coclusion} gives a conclusion of the project and finally, section \ref{sec:course-integration} gives an overview of the course integration with the different courses in the first and second year of the Bachelor in Engineering Science.

%Section \ref{sec:mechanical-design} describes the final mechanical design of the parking garage, the considered design alternatives and the motivation for the current design. The software design is described in Section \ref{sec:software-design}. It is broken down into three parts, which represent the three main components of the \ac{ips}: the Raspberry Pi software, the frontend application and the backend. Then, Section \ref{sec:user-experience} will describe a sample user experience in the \ac{ips}. Sections \ref{sec:budget} and \ref{sec:planning} describe practical aspects of the course of the project, namely the budget and the planning respectively. Finally, Section \ref{sec:course-integration} gives an overview of the course integration with the different courses in the first and second year of the Bachelor in Engineering Science.

\subsection{Problem description}\label{sec:Problem description}
% Describe the problem shortly and states which features are invaluable in the parking garage.
The official problem description is very broad: ``design a fully functional intelligent parking garage'' with the following requirements \cite{project_description}:
\begin{enumerate}
    \item the parking garage detects the amount of available parking lots;
    \item the amount of available parking lots is displayed across multiple screens;
    \item drivers can reserve a parking lot;
    \item the parking garage detects entering and exiting vehicles which eliminates the need of parking tickets.
\end{enumerate}
Therefore, the following infrastructure has to be designed, provided and built:
\begin{enumerate}
    \item sensors to detect the occupancy of a parking lot;
    \item a central server which stores and provides all the necessary data;
    \item a frontend application which the clients can use.
\end{enumerate}
The main research question this project tries to answer is ``\textit{How can we realise a safe \ac{iot} -infrastructure which makes parking easier, faster and foremost, safer?}'' \cite{project_description}.