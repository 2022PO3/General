\section{Conclusion}\label{sec:coclusion}
%The work of the previous weeks has led to a solid foundation on which can be built upon in the upcoming weeks. There's a concrete design of the major parts of the system, namely the frontend application, the backend server and the Raspberry Pi. Furthermore, a scale model of the parking garage has already been realised. All major functionalities of the different systems have been designed and connected to each other in theory. The main work of the forthcoming weeks is bringing the theory into practise and realising all the details of the different systems. The design will offer an answer to the problem of making a safe infrastructure that makes parking easier and faster.
The main goal of this project was to design an \ac{iot}-infrastructure which makes parking easier, faster and foremost safer. The main difficulty of designing this infrastructure, was the requirement of it being a safe system to park a car without violating the users privacy.\\

The final design exists out of the frontend, backend and a miniature design of a parking garage. Whereas the frontend exists out of a website and mobile application that is written with Flutter. To use these applications it is not necessary to have an account or having the app installed on the user's mobile device, which makes it easy to use for everyone. Next, the backend provides communication between the frontend application and the parking garage. For the purpose of this project it ran on a local computer. Finally the miniature design of the parking garage consists of \ac{mdf}-plates and electronic equipment, namely cameras -- needed for license plate recognition --, \acp{udms} and \acp{led}. \\

The main difficulty and priority in creating this system, was its safety. The first problem that had to be solved was with cybersecurity in the frontend. To make sure that a users account could not be easily hacked, was making sure that the user had a strong password and that he/she enabled \ac{2fa}. In the backend there were challenges to make sure that not everyone was able to access the data of different users or other vulnerable data that is stored in the backend. This issue was solved by implementing tokens. Of course there were many other problems that had to be solved, but one of the biggest problems was how to prove that a license plate belonged to one unique account. In reality this could be solved by uploading a registration certificate of the car.
\\

This project was very instructive and broad. There were things like app development, programming, working with image recognition, electronics, etc. that had to be taught. Further there was a lot of brainstorming needed for solving problems that would occur in reality with a parking garage. For example if someone doesn't show up in a garage. Although it was a very teachfull project, it was very challenging to let the frontend, backend and the parking garage work and communicate together.
